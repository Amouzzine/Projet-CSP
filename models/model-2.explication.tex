\documentclass{article}
\usepackage{amsmath, amssymb}
\usepackage{enumitem}

\begin{document}

\section*{Mod\'elisation ensembliste : model-2}

Ici la seule différence avec le model-1 réside dans la 2éme et 3éme contrainte, ici on va utiliser les opérateurs ensemblistes, et plus précisément le cardinal (|...|) d'une intersection (∩) entre des ensembles mais aussi plus d'opérations arithmétiques (produits et sommes) que ce qu'on a dans la version précédente.

\subsection*{Contrainte 3 : Deux golfeurs ne se rencontrent pas plus d'une fois}

\subsubsection*{Formulation mathématique}
\[
\forall q_1,q_2 \in \text{Golfers}, q_1 < q_2 : \sum_{w \in \text{Weeks}} \sum_{g \in \text{Groups}} (|S_{w,g} \cap \{q_1,q_2\}| \times |S_{w,g} \cap \{q_1\}| \times |S_{w,g} \cap \{q_2\}|) \leq 2
\]

\subsubsection*{Explication détaillée}
Pour comprendre cette contrainte, analysons chaque terme :

\begin{itemize}
   \item $|S_{w,g} \cap \{q_1,q_2\}|$ vaut :
   \begin{itemize}
       \item 2 si $q_1$ et $q_2$ sont dans $S_{w,g}$
       \item 1 si un seul est dans $S_{w,g}$
       \item 0 si aucun n'est dans $S_{w,g}$
   \end{itemize}
   
   \item $|S_{w,g} \cap \{q_1\}|$ vaut :
   \begin{itemize}
       \item 1 si $q_1$ est dans $S_{w,g}$
       \item 0 sinon
   \end{itemize}
   
   \item $|S_{w,g} \cap \{q_2\}|$ vaut :
   \begin{itemize}
       \item 1 si $q_2$ est dans $S_{w,g}$
       \item 0 sinon
   \end{itemize}
\end{itemize}

Le produit de ces trois termes vaut donc :
\begin{itemize}
   \item 2 si $q_1$ et $q_2$ sont dans le même groupe
   \item 0 sinon
\end{itemize}

\subsubsection*{Note sur la condition $q_1 < q_2$}
La condition $q_1 < q_2$ est utilisée pour éviter la redondance dans la vérification des paires de golfeurs. Puisque la contrainte est symétrique, vérifier la paire $(q_1,q_2)$ ou $(q_2,q_1)$ produirait le même résultat. Cette condition permet donc d'optimiser le nombre de vérifications nécessaires.

\subsection*{Contrainte 2 : L'union des groupes de chaque semaine contient tous les golfeurs}
   \[ \forall w \in [1..W] : |\bigcup_{g \in Groups} S_{wg}| = Q \]
   \begin{itemize}[label=\textendash]
       \item $\bigcup_{g \in Groups} S_{wg}$ : Union de tous les groupes de la semaine $w$
       \item $|\dots| = Q$ : Le cardinal de cette union doit être exactement $Q$ (nombre total de golfeurs)
       \item Cette contrainte vérifie que chaque semaine, tous les golfeurs participent
       \item Note: Elle n'empêche pas qu'un golfeur soit dans plusieurs groupes la même semaine
   \end{itemize}
   
\end{document}
